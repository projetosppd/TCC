
%CLASSE------------------------------------------------------------------------

% verso e anverso:
\documentclass[12pt,openright,twoside,a4paper,english,french,spanish,brazil]{abntex2}

% apenas verso:	
% \documentclass[12pt,oneside,a4paper,english,french,spanish,brazil]{abntex2} 


%PACOTES-----------------------------------------------------------------------

% ---
% Pacotes fundamentais 
% ---
\usepackage{cmap}				         % Mapear caracteres especiais no PDF
\usepackage{lmodern}			       % Usa a fonte Latin Modern			
\usepackage[T1]{fontenc}		     % Selecao de codigos de fonte.
\usepackage[utf8]{inputenc}		   % Codificacao do documento (conversão automática dos acentos)
\usepackage{lastpage}			       % Usado pela Ficha catalográfica
\usepackage{indentfirst}		     % Indenta o primeiro parágrafo de cada seção.
\usepackage{color}				       % Controle das cores
\usepackage{graphicx}            % Inclusão de gráficos
% ---

% ---
% Demais pacotes
% ---
\usepackage{ae}                  % Melhora as fontes com latin1 e fontenc
\usepackage{aecompl}             % Complemento do {ae}
\usepackage{amsmath, amsfonts, amssymb, amsthm} % Mat. avançada
\usepackage{multirow}            % Para criar tabelas
%\usepackage{subcaption}          % NÃO PODE JUNTO COM MEMOIR Criar figuras com a) b) c) etc
% usar http://tex.stackexchange.com/questions/48837/subcaption-package-incompatible-with-memoir-class
\usepackage{microtype}           % PDF fica melhor
\usepackage{array}               % Tabelas
\newcolumntype{L}{>{\centering\arraybackslash}m{0.55\linewidth}}

% ---
% Pacotes de citações
% ---
\usepackage[brazilian,hyperpageref]{backref} % Paginas com as citações na bibl
\usepackage[alf]{abntex2cite}	   % Citações padrão ABNT

% --- 
% CONFIGURAÇÕES DE PACOTES
% --- 

% ---
% Configurações do pacote backref
% Usado sem a opção hyperpageref de backref
\renewcommand{\backrefpagesname}{Citado na(s) página(s):~}
% Texto padrão antes do número das páginas
\renewcommand{\backref}{}
% Define os textos da citação
\renewcommand*{\backrefalt}[4]{
	\ifcase #1 %
		Nenhuma citação no texto.%
	\or
		Citado na página #2.%
	\else
		Citado #1 vezes nas páginas #2.%
	\fi}%
% ---

% ---
% Configurações dos pacotes matemáticos
% Definindo um operador para o seno
\newcommand{\sen}{\operatorname{sen}}
% ---

% ---
% Informações de dados para CAPA e FOLHA DE ROSTO
% ---
\titulo{Redes de Sensores Isolados Conectados via\\ Banco de Dados e Redes Neurais}
\autor{Fulano de Tal}
\local{Alegrete, RS}
\data{06 de Novembro de 2014}
\orientador{Prof. Dr. João Silva}
\coorientador{}
\instituicao{%
  Universidade Federal do Pampa -- Unipampa
  \par
  Curso de Engenharia Elétrica
  %\par
  %Programa de Pós-Graduação
  }
\tipotrabalho{Trabalho de Conclusão de Curso (Bacharelado)}
% O preambulo deve conter o tipo do trabalho, o objetivo, 
% o nome da instituição e a área de concentração 
\preambulo{Trabalho de Conclusão de Curso apresentado ao Curso de Graduação em
  Engenharia Elétrica, Área de Concentração em Controle de Sistemas, da Universidade Federal
  do Pampa (Unipampa, RS), como requisito parcial para obtenção do grau de
  \textbf{Bacharel em Engenharia Elétrica}.}
% ---


% ---
% Configurações de aparência do PDF final

% alterando o aspecto da cor azul
\definecolor{blue}{RGB}{41,5,195}

% informações do PDF
\makeatletter
\hypersetup{
     	%pagebackref=true,
		pdftitle={\@title}, 
		pdfauthor={\@author},
    	pdfsubject={\imprimirpreambulo},
	    pdfcreator={LaTeX with abnTeX2},
		pdfkeywords={Sensores}{Redes Neurais}{Conexão de Sensores}
      {Banco de Dados}, 
		colorlinks=true,       		% false: boxed links; true: colored links
    	linkcolor=blue,          	% color of internal links
    	citecolor=blue,        		% color of links to bibliography
    	filecolor=magenta,      		% color of file links
		urlcolor=blue,
		bookmarksdepth=4
}
\makeatother
% --- 

% --- 
% Espaçamentos entre linhas e parágrafos 
% --- 

% O tamanho do parágrafo é dado por:
\setlength{\parindent}{1.3cm}

% Controle do espaçamento entre um parágrafo e outro:
\setlength{\parskip}{0.2cm}  % tente também \onelineskip

% ---
% compila o indice
% ---
\makeindex
% ---


%TEXTO-------------------------------------------------------------------------

\begin{document}

% Retira espaço extra obsoleto entre as frases.
\frenchspacing 

% ----------------------------------------------------------
% ELEMENTOS PRÉ-TEXTUAIS
% ----------------------------------------------------------
% \pretextual

% ---
% Capa
% ---
\imprimircapa
% ---

% ---
% Folha de rosto
% (o * indica que haverá a ficha bibliográfica)
% ---
\imprimirfolhaderosto*
% ---

% ---
% Inserir a ficha bibliografica
% ---

% Isto é um exemplo de Ficha Catalográfica, ou ``Dados internacionais de
% catalogação-na-publicação''. Você pode utilizar este modelo como referência. 
% Porém, provavelmente a biblioteca da sua universidade lhe fornecerá um PDF
% com a ficha catalográfica definitiva após a defesa do trabalho. Quando estiver
% com o documento, salve-o como PDF no diretório do seu projeto e substitua todo
% o conteúdo de implementação deste arquivo pelo comando abaixo:
%
% \begin{fichacatalografica}
%     \includepdf{fig_ficha_catalografica.pdf}
% \end{fichacatalografica}
\begin{fichacatalografica}	
  %===============================================================================
% Ficha Catalográfica
%===============================================================================

\vspace*{\fill}					% Posição vertical
\hrule							% Linha horizontal
\begin{center}					% Minipage Centralizado
	\begin{minipage}[c]{12.5cm}		% Largura
	
		\imprimirautor
	
		\hspace{0.5cm} \imprimirtitulo  / \imprimirautor. --
		\imprimirlocal, \imprimirdata-
	
		\hspace{0.5cm} \pageref{LastPage} p. : il. (algumas color.) ; 30 cm.\\
	
		\hspace{0.5cm} \imprimirorientadorRotulo~\imprimirorientador\\
	
		\hspace{0.5cm}
		\parbox[t]{\textwidth}{\imprimirtipotrabalho~--~\imprimirinstituicao,
		\imprimirdata.}\\
	
		\hspace{0.5cm}
			1. Redes de Sensores.
			2. Redes Neurais.
			I. Prof. Dr. João da Silva.
			II. Universidade Federal do Pampa.
			III. Curso de Engenharia Elétrica.
			IV. Título\\ 			
	
		\hspace{8.75cm} CDU 00:000:000.0\\

	\end{minipage}
\end{center}
\hrule

\end{fichacatalografica}
% ---

% ---
% Inserir folha de aprovação
% ---

% Isto é um exemplo de Folha de aprovação, elemento obrigatório da NBR
% 14724/2011 (seção 4.2.1.3). Você pode utilizar este modelo até a aprovação
% do trabalho. Após isso, substitua todo o conteúdo deste arquivo por uma
% imagem da página assinada pela banca com o comando abaixo:
%
% \includepdf{folhadeaprovacao_final.pdf}
%
\begin{folhadeaprovacao}
  %===============================================================================
% Folha de Aprovação
%===============================================================================

\begin{center}
	{\ABNTEXchapterfont\large\imprimirautor}

    \vspace*{\fill}\vspace*{\fill}
    {\ABNTEXchapterfont\bfseries\Large\imprimirtitulo}
    \vspace*{\fill}
    
    \hspace{.45\textwidth}
    \begin{minipage}{.5\textwidth}
        \imprimirpreambulo
    \end{minipage}%
    \vspace*{\fill}
\end{center}
    
Trabalho aprovado. \imprimirlocal, 10 de dezembro de 2013:

\assinatura{\textbf{\imprimirorientador} \\ Orientador} 
\assinatura{\textbf{Dr. Sheldon Cooper} \\ UFSM}
\assinatura{\textbf{Dr. Anakin Skywalker} \\ Unipampa}
%\assinatura{\textbf{Professor} \\ Convidado 3}
%\assinatura{\textbf{Professor} \\ Convidado 4}
      
\begin{center}
	\vspace*{0.5cm}
    {\large\imprimirlocal}
    \par
    {\large\imprimirdata}
    \vspace*{1cm}
\end{center}
  
\end{folhadeaprovacao}
% ---

% ---
% Dedicatória
% ---
\begin{dedicatoria}   
  %===============================================================================
% Dedicatória
%===============================================================================

\vspace*{\fill}
\centering
\noindent
\textit{Dedico este trabalho a George Lucas,\\
	pela inspiração nas horas difíceis.}
\vspace*{\fill}

\end{dedicatoria}
% ---

% ---
% Agradecimentos
% ---
\begin{agradecimentos}
  Agradeço aos meu amigos, ... Agradeço a minha família, em especial aos meus pais, ...
Agradeço aos colegas de laboratório, ...
\end{agradecimentos}
% ---

% ---
% Epígrafe
% ---
\begin{epigrafe}
  %===============================================================================
% Epígrafe
%===============================================================================

\vspace*{\fill}
	\begin{flushright}
		\textit{``A velocidade da luz é um limite absoluto ao qual todas as coisas\\
			do universo estão submetidas.\\
			(Albert Einstein)}
	\end{flushright}

\end{epigrafe}
% ---

% ---
% RESUMOS
% ---

% resumo em português
\begin{resumo} 
 \input{pretextuais/resumo}
\end{resumo}

% resumo em inglês
\begin{resumo}[Abstract]
 \begin{otherlanguage*}{english}
  \input{pretextuais/abstract}
 \end{otherlanguage*}
\end{resumo}
% ---

% ---
% inserir lista de ilustrações
% ---
\pdfbookmark[0]{\listfigurename}{lof}
\listoffigures*
\cleardoublepage
% ---

% ---
% inserir lista de tabelas
% ---
\pdfbookmark[0]{\listtablename}{lot}
\listoftables*
\cleardoublepage
% ---

% ---
% inserir lista de abreviaturas e siglas
% ---
%\begin{siglas}
%  \item[Fig.] Area of the $i^{th}$ component
%  \item[456] Isto é um número
%  \item[123] Isto é outro número
%  \item[Marcelo Durgante] este é o meu nome
%\end{siglas}
% ---

% ---
% inserir lista de símbolos
% ---
%\begin{simbolos}
%  \item[$ \Gamma $] Letra grega Gama
%  \item[$ \Lambda $] Lambda
%  \item[$ \zeta $] Letra grega minúscula zeta
%  \item[$ \in $] Pertence
%\end{simbolos}
% ---

% ---
% inserir o sumario
% ---
\pdfbookmark[0]{\contentsname}{toc}
\tableofcontents*
\cleardoublepage
% ---



% ----------------------------------------------------------
% ELEMENTOS TEXTUAIS
% ----------------------------------------------------------
\textual

% ----------------------------------------------------------
% Introdução
% ----------------------------------------------------------
\chapter*[Introdução]{Introdução}
\addcontentsline{toc}{chapter}{Introdução}

\input{textuais/introducao}


% ----------------------------------------------------------
% PARTE - preparação da pesquisa
% ----------------------------------------------------------
%\part{Preparação da pesquisa}

% ----------------------------------------------------------
% Parte de revisão de literatura
% ----------------------------------------------------------

%TITULO------------------------------------------------------------------------

%==============================================================================
\chapter{Desenvolvimento}\label{sec:desenvolvimento}
%==============================================================================

	Lorem ipsum dolor sit amet, consectetur adipisicing elit, sed do eiusmod
    tempor incididunt ut labore et dolore magna aliqua. Ut enim ad minim veniam,
    quis nostrud exercitation ullamco laboris nisi ut aliquip ex ea commodo
    consequat. Duis aute irure dolor in reprehenderit in voluptate velit esse
    cillum dolore eu fugiat nulla pariatur. Excepteur sint occaecat cupidatat non
    proident, sunt in culpa qui officia deserunt mollit anim id est laborum.
    Lorem ipsum dolor sit amet, consectetur adipisicing elit, sed do eiusmod
    tempor incididunt ut labore et dolore magna aliqua. Ut enim ad minim veniam,
    quis nostrud exercitation ullamco laboris nisi ut aliquip ex ea commodo
    consequat. Duis aute irure dolor in reprehenderit in voluptate velit esse
    cillum dolore eu fugiat nulla pariatur \cite{ref:VELOSO}. Excepteur sint occaecat cupidatat non
    proident, sunt in culpa qui officia deserunt mollit anim id est laborum.

    Lorem ipsum dolor sit amet, consectetur adipisicing elit, sed do eiusmod
    tempor incididunt ut labore et dolore magna aliqua. Ut enim ad minim veniam,
    quis nostrud exercitation ullamco laboris nisi ut aliquip ex ea commodo
    consequat. Duis aute irure dolor in reprehenderit in voluptate velit esse
    cillum dolore eu fugiat nulla pariatur. Excepteur sint occaecat cupidatat non
    proident, sunt in culpa qui officia deserunt mollit anim id est laborum.
    Lorem ipsum dolor sit amet, consectetur adipisicing elit, sed do eiusmod
    tempor incididunt ut labore et dolore magna aliqua. Ut enim ad minim veniam,
    quis nostrud exercitation ullamco laboris nisi ut aliquip ex ea commodo
    consequat. Duis aute irure dolor in reprehenderit in voluptate velit esse
    cillum dolore eu fugiat nulla pariatur. Excepteur sint occaecat cupidatat non
    proident, sunt in culpa qui officia deserunt mollit anim id est laborum.
    Lorem ipsum dolor sit amet, consectetur adipisicing elit, sed do eiusmod
    tempor incididunt ut labore et dolore magna aliqua. Ut enim ad minim veniam,
    quis nostrud exercitation ullamco laboris nisi ut aliquip ex ea commodo
    consequat. Duis aute irure dolor in reprehenderit in voluptate velit esse
    cillum dolore eu fugiat nulla pariatur. Excepteur sint occaecat cupidatat non
    proident, sunt in culpa qui officia deserunt mollit anim id est laborum.

    \begin{figure}[htb]
        \centering{
            \includegraphics[width=0.5\textwidth]{figuras/logo}}
        \renewcommand\figurename{Fig.}
        \caption{Exemplo de redes de sensores.}
        \label{fig:redes_de_sensores}
    \end{figure}

    Percebe-se que a Fig. (\ref{fig:redes_de_sensores}) demonstra claramente
    o logo da Unipampa, logo a legenda da figura é uma mentira. Lorem ipsum dolor sit amet, consectetur adipisicing elit, sed do eiusmod
    tempor incididunt ut labore et dolore magna aliqua. Ut enim ad minim veniam,
    quis nostrud exercitation ullamco laboris nisi ut aliquip ex ea commodo
    consequat. Duis aute irure dolor in reprehenderit in voluptate velit esse
    cillum dolore eu fugiat nulla pariatur. Excepteur sint occaecat cupidatat non
    proident, sunt in culpa qui officia deserunt mollit anim id est laborum.
    Lorem ipsum dolor sit amet, consectetur adipisicing elit, sed do eiusmod
    tempor incididunt ut labore et dolore magna aliqua. Ut enim ad minim veniam,
    quis nostrud exercitation ullamco laboris nisi ut aliquip ex ea commodo
    consequat. Duis aute irure dolor in reprehenderit in voluptate velit esse
    cillum dolore eu fugiat nulla pariatur. Excepteur sint occaecat cupidatat non
    proident, sunt in culpa qui officia deserunt mollit anim id est laborum.

    \begin{equation}
        \begin{split}
            Z_i & = r_1 + L_1 s \\
            Z_C & = \frac{1}{s C} \\
            Z_g & = r_2 + r_g + \left( L_2 + L_g \right) s
            \label{eq:multiplas_linhas}
        \end{split}
    \end{equation}

    A equação (\ref{eq:multiplas_linhas}) apresenta três equações juntas em apenas uma
    numeração, uma forma de apresentar sistemas lineares por exemplo. Lorem ipsum dolor sit amet, consectetur adipisicing elit, sed do eiusmod
    tempor incididunt ut labore et dolore magna aliqua. Ut enim ad minim veniam,
    quis nostrud exercitation ullamco laboris nisi ut aliquip ex ea commodo
    consequat. Duis aute irure dolor in reprehenderit in voluptate velit esse
    cillum dolore eu fugiat nulla pariatur. Excepteur sint occaecat cupidatat non
    proident, sunt in culpa qui officia deserunt mollit anim id est laborum.
    Lorem ipsum dolor sit amet, consectetur adipisicing elit, sed do eiusmod
    tempor incididunt ut labore et dolore magna aliqua. Ut enim ad minim veniam,
    quis nostrud exercitation ullamco laboris nisi ut aliquip ex ea commodo
    consequat. Duis aute irure dolor in reprehenderit in voluptate velit esse
    cillum dolore eu fugiat nulla pariatur. Excepteur sint occaecat cupidatat non
    proident, sunt in culpa qui officia deserunt mollit anim id est laborum.

    \section{Modelagem Matemática}

    Lorem ipsum dolor sit amet, consectetur adipisicing elit, sed do eiusmod
    tempor incididunt ut labore et dolore magna aliqua. Ut enim ad minim veniam,
    quis nostrud exercitation ullamco laboris nisi ut aliquip ex ea commodo
    consequat. Duis aute irure dolor in reprehenderit in voluptate velit esse
    cillum dolore eu fugiat nulla pariatur. Excepteur sint occaecat cupidatat non
    proident, sunt in culpa qui officia deserunt mollit anim id est laborum.

    \begin{equation}
        V = R \cdot I
        \label{eq:equacao_desconhecida}
    \end{equation}

    A equação (\ref{eq:equacao_desconhecida}) apresenta uma relação verdadeira no
    entanto desconhecida da maioria dos engenheiros. Lorem ipsum dolor sit amet, consectetur adipisicing elit, sed do eiusmod
    tempor incididunt ut labore et dolore magna aliqua. Ut enim ad minim veniam,
    quis nostrud exercitation ullamco laboris nisi ut aliquip ex ea commodo
    consequat. Duis aute irure dolor in reprehenderit in voluptate velit esse
    cillum dolore eu fugiat nulla pariatur. Excepteur sint occaecat cupidatat non
    proident, sunt in culpa qui officia deserunt mollit anim id est laborum.

    Lorem ipsum dolor sit amet, consectetur adipisicing elit, sed do eiusmod
    tempor incididunt ut labore et dolore magna aliqua. Ut enim ad minim veniam,
    quis nostrud exercitation ullamco laboris nisi ut aliquip ex ea commodo
    consequat. Duis aute irure dolor in reprehenderit in voluptate velit esse
    cillum dolore eu fugiat nulla pariatur. Excepteur sint occaecat cupidatat non
    proident, sunt in culpa qui officia deserunt mollit anim id est laborum.
    Lorem ipsum dolor sit amet, consectetur adipisicing elit, sed do eiusmod
    tempor incididunt ut labore et dolore magna aliqua \cite{ref:RIBEIRO}. Ut enim ad minim veniam,
    quis nostrud exercitation ullamco laboris nisi ut aliquip ex ea commodo
    consequat. Duis aute irure dolor in reprehenderit in voluptate velit esse
    cillum dolore eu fugiat nulla pariatur. Excepteur sint occaecat cupidatat non
    proident, sunt in culpa qui officia deserunt mollit anim id est laborum.

    \begin{figure}[htb]
        \centering{
            \includegraphics[width=0.3\textwidth]{figuras/logo}}
        \renewcommand\figurename{Fig.}
        \caption{Esta figura ocupa 30\% do tamanho da linha.}
        \label{fig:figura_pequena}
    \end{figure}

    Já a Fig. \ref{fig:figura_pequena} é uma figura pequena que ocupa pouco espaço.
    Lorem ipsum dolor sit amet, consectetur adipisicing elit, sed do eiusmod
    tempor incididunt ut labore et dolore magna aliqua. Ut enim ad minim veniam,
    quis nostrud exercitation ullamco laboris nisi ut aliquip ex ea commodo
    consequat. Duis aute irure dolor in reprehenderit in voluptate velit esse
    cillum dolore eu fugiat nulla pariatur. Excepteur sint occaecat cupidatat non
    proident, sunt in culpa qui officia deserunt mollit anim id est laborum.
    Lorem ipsum dolor sit amet, consectetur adipisicing elit, sed do eiusmod
    tempor incididunt ut labore et dolore magna aliqua. Ut enim ad minim veniam,
    quis nostrud exercitation ullamco laboris nisi ut aliquip ex ea commodo
    consequat. Duis aute irure dolor in reprehenderit in voluptate velit esse
    cillum dolore eu fugiat nulla pariatur. Excepteur sint occaecat cupidatat non
    proident, sunt in culpa qui officia deserunt mollit anim id est laborum \cite{ref:DURGANTE}.

    \begin{equation}
        E = m \cdot c^2
        \label{eq:equacao_conhecida}
    \end{equation}

    A equação (\ref{eq:equacao_conhecida}) apresenta uma relação verdadeira no
    entanto desconhecida da maioria dos engenheiros. Lorem ipsum dolor sit amet, consectetur adipisicing elit, sed do eiusmod
    tempor incididunt ut labore et dolore magna aliqua. Ut enim ad minim veniam,
    quis nostrud exercitation ullamco laboris nisi ut aliquip ex ea commodo
    consequat. Duis aute irure dolor in reprehenderit in voluptate velit esse
    cillum dolore eu fugiat nulla pariatur. Excepteur sint occaecat cupidatat non
    proident, sunt in culpa qui officia deserunt mollit anim id est laborum.

    \section{Resultados}

    Lorem ipsum dolor sit amet, consectetur adipisicing elit, sed do eiusmod
    tempor incididunt ut labore et dolore magna aliqua. Ut enim ad minim veniam,
    quis nostrud exercitation ullamco laboris nisi ut aliquip ex ea commodo
    consequat. Duis aute irure dolor in reprehenderit in voluptate velit esse
    cillum dolore eu fugiat nulla pariatur. Excepteur sint occaecat cupidatat non
    proident, sunt in culpa qui officia deserunt mollit anim id est laborum.
    Lorem ipsum dolor sit amet, consectetur adipisicing elit, sed do eiusmod
    tempor incididunt ut labore et dolore magna aliqua \cite{ref:RIBEIRO}. Ut enim ad minim veniam,
    quis nostrud exercitation ullamco laboris nisi ut aliquip ex ea commodo
    consequat. Duis aute irure dolor in reprehenderit in voluptate velit esse
    cillum dolore eu fugiat nulla pariatur. Excepteur sint occaecat cupidatat non
    proident, sunt in culpa qui officia deserunt mollit anim id est laborum.

    \begin{figure}[htb]
        \centering{
            \includegraphics[width=0.5\textwidth]{figuras/logo}}
        \renewcommand\figurename{Fig.}
        \caption{Esta figura ocupa 50\% do tamanho da linha.}
        \label{fig:figura_media}
    \end{figure}

    Já a Fig. \ref{fig:figura_media} é uma figura media que ocupa pouco espaço.
    Lorem ipsum dolor sit amet, consectetur adipisicing elit, sed do eiusmod
    tempor incididunt ut labore et dolore magna aliqua. Ut enim ad minim veniam,
    quis nostrud exercitation ullamco laboris nisi ut aliquip ex ea commodo
    consequat. Duis aute irure dolor in reprehenderit in voluptate velit esse
    cillum dolore eu fugiat nulla pariatur. Excepteur sint occaecat cupidatat non
    proident, sunt in culpa qui officia deserunt mollit anim id est laborum.
    Lorem ipsum dolor sit amet, consectetur adipisicing elit, sed do eiusmod
    tempor incididunt ut labore et dolore magna aliqua. Ut enim ad minim veniam,
    quis nostrud exercitation ullamco laboris nisi ut aliquip ex ea commodo
    consequat. Duis aute irure dolor in reprehenderit in voluptate velit esse
    cillum dolore eu fugiat nulla pariatur. Excepteur sint occaecat cupidatat non
    proident, sunt in culpa qui officia deserunt mollit anim id est laborum \cite{ref:DURGANTE}.

    \begin{equation}
        \int_a^b \! f(x) \, \mathrm{d}x = \lim_{x \to \infty} \frac{b - a}{n} \sum_{k=1}^N
            f\left( a + \frac{b - a}{n} \cdot k \right)
        \label{eq:equacao_complicada}
    \end{equation}

    A equação (\ref{eq:equacao_complicada}) apresenta uma relação verdadeira no
    entanto desconhecida da maioria dos engenheiros. Lorem ipsum dolor sit amet, consectetur adipisicing elit, sed do eiusmod
    tempor incididunt ut labore et dolore magna aliqua. Ut enim ad minim veniam,
    quis nostrud exercitation ullamco laboris nisi ut aliquip ex ea commodo
    consequat. Duis aute irure dolor in reprehenderit in voluptate velit esse
    cillum dolore eu fugiat nulla pariatur. Excepteur sint occaecat cupidatat non
    proident, sunt in culpa qui officia deserunt mollit anim id est laborum.

    \section{Algumas Coisas Relevantes}

    Só pra constar, é possível incluir tabelas.

    \begin{table}[htb]
        \renewcommand{\arraystretch}{1.35}
        \setlength{\tabcolsep}{1.2mm}
        \caption{Parâmetros do sistema utilizados no projeto.}
        \label{tab:parametros_simulacao}
        \centering
        \begin{tabular}{c c c c}
            \hline
            \multicolumn{1}{c}{Parâmetro} & \multicolumn{1}{c}{Valor} &
            \multicolumn{1}{c}{Parâmetro} & \multicolumn{1}{c}{Valor} \\
            \hline
            $\theta_1$ &  $17$      &  $k$                                   &  $12$    \\
            $\zeta$   &  $40\mu$A   & Frequência de Amostragem ($f_s = 1/T_s$) &  $400$MHz  \\
            \hline
        \end{tabular}
    \end{table}

    Incluída a Tabela \ref{tab:parametros_simulacao}, que apresenta os parametros
    do sistema utilizados para simulação. Lorem ipsum dolor sit amet, consectetur adipisicing elit, sed do eiusmod
    tempor incididunt ut labore et dolore magna aliqua. Ut enim ad minim veniam,
    quis nostrud exercitation ullamco laboris nisi ut aliquip ex ea commodo
    consequat. Duis aute irure dolor in reprehenderit in voluptate velit esse
    cillum dolore eu fugiat nulla pariatur. Excepteur sint occaecat cupidatat non
    proident, sunt in culpa qui officia deserunt mollit anim id est laborum.
    Lorem ipsum dolor sit amet, consectetur adipisicing elit, sed do eiusmod
    tempor incididunt ut labore et dolore magna aliqua \cite{ref:RIBEIRO}. Ut enim ad minim veniam,
    quis nostrud exercitation ullamco laboris nisi ut aliquip ex ea commodo
    consequat. Duis aute irure dolor in reprehenderit in voluptate velit esse
    cillum dolore eu fugiat nulla pariatur. Excepteur sint occaecat cupidatat non
    proident, sunt in culpa qui officia deserunt mollit anim id est laborum.

    \begin{figure}[htb]
        \centering{
            \includegraphics[width=0.7\textwidth, height= 3cm]{figuras/logo}}
        \renewcommand\figurename{Fig.}
        \caption{Esta figura ocupa 70\% do tamanho da linha.}
        \label{fig:figura_media}
    \end{figure}

    Já a Fig. \ref{fig:figura_media} é uma figura media que ocupa pouco espaço.
    Lorem ipsum dolor sit amet, consectetur adipisicing elit, sed do eiusmod
    tempor incididunt ut labore et dolore magna aliqua. Ut enim ad minim veniam,
    quis nostrud exercitation ullamco laboris nisi ut aliquip ex ea commodo
    consequat. Duis aute irure dolor in reprehenderit in voluptate velit esse
    cillum dolore eu fugiat nulla pariatur.

    \section{Algumas Coisas Ainda Mais Interessantes}

    Excepteur sint occaecat cupidatat non
    proident, sunt in culpa qui officia deserunt mollit anim id est laborum.
    Lorem ipsum dolor sit amet, consectetur adipisicing elit, sed do eiusmod
    tempor incididunt ut labore et dolore magna aliqua. Ut enim ad minim veniam,
    quis nostrud exercitation ullamco laboris nisi ut aliquip ex ea commodo
    consequat. Duis aute irure dolor in reprehenderit in voluptate velit esse
    cillum dolore eu fugiat nulla pariatur. Excepteur sint occaecat cupidatat non
    proident, sunt in culpa qui officia deserunt mollit anim id est laborum \cite{ref:DURGANTE}.

    Lorem ipsum dolor sit amet, consectetur adipisicing elit, sed do eiusmod
    tempor incididunt ut labore et dolore magna aliqua. Ut enim ad minim veniam,
    quis nostrud exercitation ullamco laboris nisi ut aliquip ex ea commodo
    consequat. Duis aute irure dolor in reprehenderit in voluptate velit esse
    cillum dolore eu fugiat nulla pariatur. Excepteur sint occaecat cupidatat non
    proident, sunt in culpa qui officia deserunt mollit anim id est laborum.

    Lorem ipsum dolor sit amet, consectetur adipisicing elit, sed do eiusmod
    tempor incididunt ut labore et dolore magna aliqua. Ut enim ad minim veniam,
    quis nostrud exercitation ullamco laboris nisi ut aliquip ex ea commodo
    consequat. Duis aute irure dolor in reprehenderit in voluptate velit esse
    cillum dolore eu fugiat nulla pariatur. Excepteur sint occaecat cupidatat non
    proident, sunt in culpa qui officia deserunt mollit anim id est laborum.
    Lorem ipsum dolor sit amet, consectetur adipisicing elit, sed do eiusmod
    tempor incididunt ut labore et dolore magna aliqua. Ut enim ad minim veniam,
    quis nostrud exercitation ullamco laboris nisi ut aliquip ex ea commodo
    consequat. Duis aute irure dolor in reprehenderit in voluptate velit esse
    cillum dolore eu fugiat nulla pariatur. Excepteur sint occaecat cupidatat non
    proident, sunt in culpa qui officia deserunt mollit anim id est laborum.

    Lorem ipsum dolor sit amet, consectetur adipisicing elit, sed do eiusmod
    tempor incididunt ut labore et dolore magna aliqua. Ut enim ad minim veniam,
    quis nostrud exercitation ullamco laboris nisi ut aliquip ex ea commodo
    consequat. Duis aute irure dolor in reprehenderit in voluptate velit esse
    cillum dolore eu fugiat nulla pariatur. Excepteur sint occaecat cupidatat non
    proident, sunt in culpa qui officia deserunt mollit anim id est laborum.

    \subsection{Algumas Coisas SubInteressantes}

    Lorem ipsum dolor sit amet, consectetur adipisicing elit, sed do eiusmod
    tempor incididunt ut labore et dolore magna aliqua. Ut enim ad minim veniam,
    quis nostrud exercitation ullamco laboris nisi ut aliquip ex ea commodo
    consequat. Duis aute irure dolor in reprehenderit in voluptate velit esse
    cillum dolore eu fugiat nulla pariatur. Excepteur sint occaecat cupidatat non
    proident, sunt in culpa qui officia deserunt mollit anim id est laborum.
    Lorem ipsum dolor sit amet, consectetur adipisicing elit, sed do eiusmod
    tempor incididunt ut labore et dolore magna aliqua. Ut enim ad minim veniam,
    quis nostrud exercitation ullamco laboris nisi ut aliquip ex ea commodo
    consequat. Duis aute irure dolor in reprehenderit in voluptate velit esse
    cillum dolore eu fugiat nulla pariatur. Excepteur sint occaecat cupidatat non
    proident, sunt in culpa qui officia deserunt mollit anim id est laborum.

    \section{Objetivos}

    Uma lista de objetivos para este trabalho:
    \begin{itemize}
        \item Este é o primeiro objetivo;
        \item b) é possível numerar as coisas manualmente;
        \item c) no entanto, para numerar coisas, sugere-se usar outra estrutura.
    \end{itemize}


    Agora uma lista numerada:
    \begin{enumerate}
        \item Uma estrutura como esta!
        \item Este é o segundo item;
        \item É possível mudar o marcador de numeração;
        \item É possível por letras, números, algarismos romanos, etc.
    \end{enumerate}


%FIM---------------------------------------------------------------------------


% ----------------------------------------------------------
% Resultados
% ----------------------------------------------------------
%\part{Resultados}


% ---
% Finaliza a parte no bookmark do PDF, para que se inicie o bookmark na raiz
% ---
\bookmarksetup{startatroot}% 
% ---

% ---
% Conclusão
% ---
\chapter*[Conclusão]{Conclusão}
\addcontentsline{toc}{chapter}{Conclusão}

%TITULO------------------------------------------------------------------------

%==============================================================================
\label{sec:conclusao}
%==============================================================================

	A seção \ref{sec:introducao} apresentou os conceitos iniciais e o estado da
    arte para este trabalho. A seção \ref{sec:desenvolvimento} apresentou outras
    coisas muito legais. Nesta seção, serão tecidos comentários sobre os resultados
    apresentados neste trabalho.

    Lorem ipsum dolor sit amet, consectetur adipisicing elit, sed do eiusmod
    tempor incididunt ut labore et dolore magna aliqua. Ut enim ad minim veniam,
    quis nostrud exercitation ullamco laboris nisi ut aliquip ex ea commodo
    consequat. Duis aute irure dolor in reprehenderit in voluptate velit esse
    cillum dolore eu fugiat nulla pariatur. Excepteur sint occaecat cupidatat non
    proident, sunt in culpa qui officia deserunt mollit anim id est laborum.
    Lorem ipsum dolor sit amet, consectetur adipisicing elit, sed do eiusmod
    tempor incididunt ut labore et dolore magna aliqua. Ut enim ad minim veniam,
    quis nostrud exercitation ullamco laboris nisi ut aliquip ex ea commodo
    consequat. Duis aute irure dolor in reprehenderit in voluptate velit esse
    cillum dolore eu fugiat nulla pariatur. Excepteur sint occaecat cupidatat non
    proident, sunt in culpa qui officia deserunt mollit anim id est laborum.

    Lorem ipsum dolor sit amet, consectetur adipisicing elit, sed do eiusmod
    tempor incididunt ut labore et dolore magna aliqua. Ut enim ad minim veniam,
    quis nostrud exercitation ullamco laboris nisi ut aliquip ex ea commodo
    consequat. Duis aute irure dolor in reprehenderit in voluptate velit esse
    cillum dolore eu fugiat nulla pariatur. Excepteur sint occaecat cupidatat non
    proident, sunt in culpa qui officia deserunt mollit anim id est laborum.

    Lorem ipsum dolor sit amet, consectetur adipisicing elit, sed do eiusmod
    tempor incididunt ut labore et dolore magna aliqua. Ut enim ad minim veniam,
    quis nostrud exercitation ullamco laboris nisi ut aliquip ex ea commodo
    consequat. Duis aute irure dolor in reprehenderit in voluptate velit esse
    cillum dolore eu fugiat nulla pariatur. Excepteur sint occaecat cupidatat non
    proident, sunt in culpa qui officia deserunt mollit anim id est laborum.

    Lorem ipsum dolor sit amet, consectetur adipisicing elit, sed do eiusmod
    tempor incididunt ut labore et dolore magna aliqua. Ut enim ad minim veniam,
    quis nostrud exercitation ullamco laboris nisi ut aliquip ex ea commodo
    consequat. Duis aute irure dolor in reprehenderit in voluptate velit esse
    cillum dolore eu fugiat nulla pariatur. Excepteur sint occaecat cupidatat non
    proident, sunt in culpa qui officia deserunt mollit anim id est laborum.
    Lorem ipsum dolor sit amet, consectetur adipisicing elit, sed do eiusmod
    tempor incididunt ut labore et dolore magna aliqua. Ut enim ad minim veniam,
    quis nostrud exercitation ullamco laboris nisi ut aliquip ex ea commodo
    consequat. Duis aute irure dolor in reprehenderit in voluptate velit esse
    cillum dolore eu fugiat nulla pariatur. Excepteur sint occaecat cupidatat non
    proident, sunt in culpa qui officia deserunt mollit anim id est laborum.
    Lorem ipsum dolor sit amet, consectetur adipisicing elit, sed do eiusmod
    tempor incididunt ut labore et dolore magna aliqua. Ut enim ad minim veniam,
    quis nostrud exercitation ullamco laboris nisi ut aliquip ex ea commodo
    consequat. Duis aute irure dolor in reprehenderit in voluptate velit esse
    cillum dolore eu fugiat nulla pariatur. Excepteur sint occaecat cupidatat non
    proident, sunt in culpa qui officia deserunt mollit anim id est laborum.

    Lorem ipsum dolor sit amet, consectetur adipisicing elit, sed do eiusmod
    tempor incididunt ut labore et dolore magna aliqua. Ut enim ad minim veniam,
    quis nostrud exercitation ullamco laboris nisi ut aliquip ex ea commodo
    consequat. Duis aute irure dolor in reprehenderit in voluptate velit esse
    cillum dolore eu fugiat nulla pariatur. Excepteur sint occaecat cupidatat non
    proident, sunt in culpa qui officia deserunt mollit anim id est laborum.

    Lorem ipsum dolor sit amet, consectetur adipisicing elit, sed do eiusmod
    tempor incididunt ut labore et dolore magna aliqua. Ut enim ad minim veniam,
    quis nostrud exercitation ullamco laboris nisi ut aliquip ex ea commodo
    consequat. Duis aute irure dolor in reprehenderit in voluptate velit esse
    cillum dolore eu fugiat nulla pariatur. Excepteur sint occaecat cupidatat non
    proident, sunt in culpa qui officia deserunt mollit anim id est laborum.
    Lorem ipsum dolor sit amet, consectetur adipisicing elit, sed do eiusmod
    tempor incididunt ut labore et dolore magna aliqua. Ut enim ad minim veniam,
    quis nostrud exercitation ullamco laboris nisi ut aliquip ex ea commodo
    consequat. Duis aute irure dolor in reprehenderit in voluptate velit esse
    cillum dolore eu fugiat nulla pariatur. Excepteur sint occaecat cupidatat non
    proident, sunt in culpa qui officia deserunt mollit anim id est laborum.
    Lorem ipsum dolor sit amet, consectetur adipisicing elit, sed do eiusmod
    tempor incididunt ut labore et dolore magna aliqua. Ut enim ad minim veniam,
    quis nostrud exercitation ullamco laboris nisi ut aliquip ex ea commodo
    consequat. Duis aute irure dolor in reprehenderit in voluptate velit esse
    cillum dolore eu fugiat nulla pariatur. Excepteur sint occaecat cupidatat non
    proident, sunt in culpa qui officia deserunt mollit anim id est laborum.

    Lorem ipsum dolor sit amet, consectetur adipisicing elit, sed do eiusmod
    tempor incididunt ut labore et dolore magna aliqua. Ut enim ad minim veniam,
    quis nostrud exercitation ullamco laboris nisi ut aliquip ex ea commodo
    consequat. Duis aute irure dolor in reprehenderit in voluptate velit esse
    cillum dolore eu fugiat nulla pariatur. Excepteur sint occaecat cupidatat non
    proident, sunt in culpa qui officia deserunt mollit anim id est laborum.
    Lorem ipsum dolor sit amet, consectetur adipisicing elit, sed do eiusmod
    tempor incididunt ut labore et dolore magna aliqua. Ut enim ad minim veniam,
    quis nostrud exercitation ullamco laboris nisi ut aliquip ex ea commodo
    consequat. Duis aute irure dolor in reprehenderit in voluptate velit esse
    cillum dolore eu fugiat nulla pariatur. Excepteur sint occaecat cupidatat non
    proident, sunt in culpa qui officia deserunt mollit anim id est laborum.
    Lorem ipsum dolor sit amet, consectetur adipisicing elit, sed do eiusmod
    tempor incididunt ut labore et dolore magna aliqua. Ut enim ad minim veniam,
    quis nostrud exercitation ullamco laboris nisi ut aliquip ex ea commodo
    consequat. Duis aute irure dolor in reprehenderit in voluptate velit esse
    cillum dolore eu fugiat nulla pariatur. Excepteur sint occaecat cupidatat non
    proident, sunt in culpa qui officia deserunt mollit anim id est laborum.

    Lorem ipsum dolor sit amet, consectetur adipisicing elit, sed do eiusmod
    tempor incididunt ut labore et dolore magna aliqua. Ut enim ad minim veniam,
    quis nostrud exercitation ullamco laboris nisi ut aliquip ex ea commodo
    consequat. Duis aute irure dolor in reprehenderit in voluptate velit esse
    cillum dolore eu fugiat nulla pariatur. Excepteur sint occaecat cupidatat non
    proident, sunt in culpa qui officia deserunt mollit anim id est laborum.

    Lorem ipsum dolor sit amet, consectetur adipisicing elit, sed do eiusmod
    tempor incididunt ut labore et dolore magna aliqua. Ut enim ad minim veniam,
    quis nostrud exercitation ullamco laboris nisi ut aliquip ex ea commodo
    consequat. Duis aute irure dolor in reprehenderit in voluptate velit esse
    cillum dolore eu fugiat nulla pariatur. Excepteur sint occaecat cupidatat non
    proident, sunt in culpa qui officia deserunt mollit anim id est laborum.
    Lorem ipsum dolor sit amet, consectetur adipisicing elit, sed do eiusmod
    tempor incididunt ut labore et dolore magna aliqua. Ut enim ad minim veniam,
    quis nostrud exercitation ullamco laboris nisi ut aliquip ex ea commodo
    consequat. Duis aute irure dolor in reprehenderit in voluptate velit esse
    cillum dolore eu fugiat nulla pariatur. Excepteur sint occaecat cupidatat non
    proident, sunt in culpa qui officia deserunt mollit anim id est laborum.
    Lorem ipsum dolor sit amet, consectetur adipisicing elit, sed do eiusmod
    tempor incididunt ut labore et dolore magna aliqua. Ut enim ad minim veniam,
    quis nostrud exercitation ullamco laboris nisi ut aliquip ex ea commodo
    consequat. Duis aute irure dolor in reprehenderit in voluptate velit esse
    cillum dolore eu fugiat nulla pariatur. Excepteur sint occaecat cupidatat non
    proident, sunt in culpa qui officia deserunt mollit anim id est laborum.

    Lorem ipsum dolor sit amet, consectetur adipisicing elit, sed do eiusmod
    tempor incididunt ut labore et dolore magna aliqua. Ut enim ad minim veniam,
    quis nostrud exercitation ullamco laboris nisi ut aliquip ex ea commodo
    consequat. Duis aute irure dolor in reprehenderit in voluptate velit esse
    cillum dolore eu fugiat nulla pariatur. Excepteur sint occaecat cupidatat non
    proident, sunt in culpa qui officia deserunt mollit anim id est laborum.

    Lorem ipsum dolor sit amet, consectetur adipisicing elit, sed do eiusmod
    tempor incididunt ut labore et dolore magna aliqua. Ut enim ad minim veniam,
    quis nostrud exercitation ullamco laboris nisi ut aliquip ex ea commodo
    consequat. Duis aute irure dolor in reprehenderit in voluptate velit esse
    cillum dolore eu fugiat nulla pariatur. Excepteur sint occaecat cupidatat non
    proident, sunt in culpa qui officia deserunt mollit anim id est laborum.
    Lorem ipsum dolor sit amet, consectetur adipisicing elit, sed do eiusmod
    tempor incididunt ut labore et dolore magna aliqua. Ut enim ad minim veniam,
    quis nostrud exercitation ullamco laboris nisi ut aliquip ex ea commodo
    consequat. Duis aute irure dolor in reprehenderit in voluptate velit esse
    cillum dolore eu fugiat nulla pariatur. Excepteur sint occaecat cupidatat non
    proident, sunt in culpa qui officia deserunt mollit anim id est laborum.
    Lorem ipsum dolor sit amet, consectetur adipisicing elit, sed do eiusmod
    tempor incididunt ut labore et dolore magna aliqua. Ut enim ad minim veniam,
    quis nostrud exercitation ullamco laboris nisi ut aliquip ex ea commodo
    consequat. Duis aute irure dolor in reprehenderit in voluptate velit esse
    cillum dolore eu fugiat nulla pariatur. Excepteur sint occaecat cupidatat non
    proident, sunt in culpa qui officia deserunt mollit anim id est laborum.

    Lorem ipsum dolor sit amet, consectetur adipisicing elit, sed do eiusmod
    tempor incididunt ut labore et dolore magna aliqua. Ut enim ad minim veniam,
    quis nostrud exercitation ullamco laboris nisi ut aliquip ex ea commodo
    consequat. Duis aute irure dolor in reprehenderit in voluptate velit esse
    cillum dolore eu fugiat nulla pariatur. Excepteur sint occaecat cupidatat non
    proident, sunt in culpa qui officia deserunt mollit anim id est laborum.
    
    Portanto, conclui-se que nada se pode concluir sobre o limite.


%FIM---------------------------------------------------------------------------




% ----------------------------------------------------------
% ELEMENTOS PÓS-TEXTUAIS
% ----------------------------------------------------------
\postextual


% ----------------------------------------------------------
% Referências bibliográficas
% ----------------------------------------------------------
\bibliography{tcc}

% ----------------------------------------------------------
% Glossário
% ----------------------------------------------------------
%
% Consulte o manual da classe abntex2 para orientações sobre o glossário.
%
%\glossary

% ----------------------------------------------------------
% Apêndices
% ----------------------------------------------------------

% ---
% Inicia os apêndices
% ---
%\begin{apendicesenv}

% Imprime uma página indicando o início dos apêndices
%\partapendices

% ----------------------------------------------------------
%\chapter{Quisque libero justo}
% ----------------------------------------------------------

%\end{apendicesenv}
% ---


% ----------------------------------------------------------
% Anexos
% ----------------------------------------------------------

% ---
% Inicia os anexos
% ---
%\begin{anexosenv}

% Imprime uma página indicando o início dos anexos
%\partanexos

%\chapter{Procedimento de Projeto do Filtro \textit{LCL}}

	O projeto de um filtro LCL pode ser feito de várias maneiras, dependendo
    do objetivo do projetista. O procedimento de projeto apresentado em~\cite{ref:TANG}
    é muito utilizado, por ser generalizado, simples e em valores \textit{por unidade},
    o que torna simples a escalabilidade do sistema. Os passos deste procedimento são:

    \begin{enumerate}

        \item Definir qual a ordem $k$ mais alta das correntes harmônicas que precisam ser
        compensadas. A frequência de ressonância $\omega_{res}$ deve ser função da frequência
        fundamental nominal $\omega_n$:

        \begin{equation}
            \frac{k \omega_n}{0,3} \leq \omega_{res} \leq \frac{k \omega_n}{0,25}
        \label{eq:wres}
        \end{equation}

        \item A frequência de comutação deve ser pelo menos duas vezes maior que a frequência
        de ressonância. Valores maiores podem ser usados para uma melhor atenuação harmônica,
        mas resultarão em mais perdas.

        \item Valores de impedância, capacitância e indutância base devem ser definidos. Dessa
        forma, a impedância base $Z_b$ é função da tensão nominal $V$ e da potência nominal $P$:

        \begin{equation}
            Z_b = \frac{V^2}{P}
        \end{equation}

        Os valores da capacitância e indutância base são, respectivamente:

        \begin{equation}
            C_b = \frac{1}{\omega_n Z_b}
        \end{equation}

        \begin{equation}
            L_b = \frac{Z_b}{\omega_n}
        \end{equation}

        \item As indutâncias do lado do conversor $L_{ff}$ e da rede $L_{fg}$ devem ser iguais para
        produzir a menor frequência de ressonância possível, e a máxima atenuação de harmônicas de
        comutação. Além disso, é recomendável que o valor total em \textit{por unidade} dos dois
        indutores seja igual ao valor do capacitor do filtro $C_f$. Desta forma:

        \begin{equation}
            L_{ff} = L_{gf} = \frac{1}{4k} L_b
        \end{equation}

        \begin{equation}
            C_f = \frac{1}{2k} C_b
        \label{eq:cf}
        \end{equation}

        \item O valor comercial de capacitor mais próximo ao valor encontrado em (\ref{eq:cf})
        deve ser escolhido, e os valores de indutância ajustados de acordo. A frequência de
        ressonância recalculada com os valores ajustados deve, no entando, estar de acordo com
        (\ref{eq:wres}).

    \end{enumerate}


% \chapter{Cronograma}
% \label{cronograma}
% \addcontentsline{toc}{chapter}{Cronograma}

% \begin{table}[htb]
%    \centering
%    \begin{small}        
%        \setlength{\tabcolsep}{3pt}%redefine espaçamento das colunas
%        \begin{tabular}{|L|c|c|c|c|c|c|c|c|}\hline
%             & \multicolumn{8}{c|}{\textbf{Meses}}\\ \cline{2-9}
%            \raisebox{1.5ex}{\textbf{Etapa}} & 02 & 03 & 04 & 05 & 06 & 07 & 08 & 09 \\ \hline
%            Qualificação 
%            & \textbullet &   &   &   &   &   &   &   \\ \hline
%            Resultados de simulação do controlador Multi-malhas com malha interna utilizando a tensão
%            do capacitor numa estrutura IMC
%            & \textbullet &   &   &   &   &   &   &   \\ \hline
%            Resultados de simulação do controlador Multi-malhas com malha interna utilizando a corrente
%            do capacitor numa estrutura IMC
%            &   & \textbullet &   &   &   &   &   &   \\ \hline
%            Resultados experimentais
%            &   & \textbullet & \textbullet &   &   &   &   &   \\ \hline
%            Escrita de um artigo científico
%            &   & \textbullet & \textbullet &   &   &   &   &   \\ \hline
%            Comparação das duas técnicas multi-malhas utilizadas, entre si e com outras técnicas clássicas
%            (PI e Deadbeat, por exemplo)
%            &   &   &   & \textbullet & \textbullet &   &   &   \\ \hline
%            Verificação de melhorias da relação Desempenho x Robustez
%            &   &   &   &   & \textbullet &   &   &   \\ \hline
%            Análise dos resultados e redação
%            &   &   &   &   &   & \textbullet & \textbullet &   \\ \hline
%            Defesa
%            &   &   &   &   &   &   &   & \textbullet \\ \hline
%        \end{tabular}
%    \end{small}
% \end{table}


%\end{anexosenv}

%---------------------------------------------------------------------
% INDICE REMISSIVO
%---------------------------------------------------------------------

\printindex

\end{document}
